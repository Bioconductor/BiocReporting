
\title{News from the Bioconductor Project}
\author{by Bioconductor Core Team}
\maketitle

\href{https://bioconductor.org}{Bioconductor} provides
tools for the analysis and comprehension of high-throughput genomic
data.  The project has entered its twentieth year, with funding
for core development and infrastructure maintenance secured
through 2025 (NIH NHGRI 2U24HG004059).  Additional support is provided
by NIH NCI, Chan-Zuckerberg Initiative, National Science Foundation,
Microsoft, and Amazon.  In this news report, we give some
details about the software and data resource collection,
infrastructure for building, checking, and distributing resources,
core team activities, and some new initiatives.
 
\textit{Software ecosystem}

Bioconductor 3.16 was released on 2 November, 2022. It is
compatible with R 4.2 and consists of 2183 software packages, 416
experiment data packages, 909 up-to-date annotation packages, 28
workflows, and 3 books. \href{https://bioconductor.org/books/release/}{Books} are
built regularly from source and therefore fully
reproducible; an example is the
community-developed \href{https://bioconductor.org/books/release/OSCA/}{Orchestrating
Single-Cell Analysis with Bioconductor}.
The Bioconductor
\href{https://bioconductor.org/news/bioc_3_16_release/}{3.16 release
  announcement} includes descriptions of 71 new software packages, 9 new data
  experiment packages, 2 new annotation packages, and updates to NEWS files for
  many additional packages. 

\textit{Infrastructure updates}

\begin{itemize}
\end{itemize}

\textit{Core team updates}
\begin{itemize}
\item New developer XXX has joined the Bioconductor Core Developer Team.  xxx is a member of xxx.
\item xxx is joined by long-term core members Lori Kern of Roswell Park
Comprehensive Cancer Center, Marcel Ramos of CUNY and Roswell, Herv\'e Pages of
Fred Hutchinson Cancer Research Center, Jennifer Wokaty of CUNY, and Alex
Mahmoud at Channing Division of Network Medicine.
\end{itemize}

\textit{New initiatives}

\begin{itemize}
\end{itemize}



\textit{Using Bioconductor}

Start using
Bioconductor by installing the most recent version of R and evaluating
the commands
\begin{example}
  if (!requireNamespace("BiocManager", quietly = TRUE))
      install.packages("BiocManager")
  BiocManager::install()
\end{example}
Install additional packages and dependencies,
e.g., \BIOpkg{SingleCellExperiment}, with
\begin{example}
  BiocManager::install("SingleCellExperiment")
\end{example}
\href{https://bioconductor.org/help/docker/}{Docker}
images provides a very effective on-ramp for power users to rapidly
obtain access to standardized and scalable computing environments.
Key resources include:
%% 
\begin{itemize}
\item \href{https://bioconductor.org}{bioconductor.org} to install,
  learn, use, and develop Bioconductor packages.
\item A list of \href{https://bioconductor.org/packages}{available
  software}, linking to pages describing each package.
\item A question-and-answer style
  \href{https://support.bioconductor.org}{user support site} and
  developer-oriented
  \href{https://stat.ethz.ch/mailman/listinfo/bioc-devel}{mailing
    list}.
\item A community slack (\href{https://bioc-community.herokuapp.com/}{sign up})
   for extended technical discussion.
\item The
  \href{https://f1000research.com/channels/bioconductor}{F1000Research
    Bioconductor channel} for peer-reviewed Bioconductor work flows.
\item The \href{https://www.youtube.com/user/bioconductor}{Bioconductor YouTube} 
     channel includes recordings of keynote and talks from recent 
     conferences including Bioc2022, EuroBioC2022, and BiocAsia2021, in addition to 
     video recordings of training courses. 
\item Our \href{https://github.com/Bioconductor/Contributions}{package
  submission} repository for open technical review of new packages.
\end{itemize}

Recent Bioconductor conferences include
\href{https://bioc2022.bioconductor.org}{BioC 2022} (July 27-29),
\href{https://eurobioc2022.bioconductor.org/}{European Bioconductor Meeting}
(September 14-16),
and \href{https://biocasia2022.bioconductor.org/}{BiocAsia2022} (December
1-2). Each had invited and contributed talks, as well as workshops and other sessions to enable community
participation. Slides, videos, and workshop material for each conference are, or
will soon be, available on each conference web site as well as from
the \href{http://bioconductor.org/help/course-materials/}{Courses and
Conferences} section of the Bioconductor web site.

The Bioconductor project continues to mature as a
community. The \href{https://bioconductor.org/about/technical-advisory-board/}{Technical}
and \href{https://bioconductor.org/about/community-advisory-board/}{Community}
Advisory Boards provide guidance to ensure that the project addresses
leading-edge biological problems with advanced technical approaches,
and adopts practices (such as a
project-wide \href{https://bioconductor.org/about/code-of-conduct/}{Code
of Conduct}) that encourages all to participate. We look forward to
welcoming you!

\address{Bioconductor Core Team \\
    Channing Division of Network Medicine \\
    Mass General Brigham \\
    Harvard Medical School, Boston, MA \\ \\
    Department of Data Science \\
    Dana-Farber Cancer Institute \\
    Harvard Medical School, Boston, MA \\ \\
    Biostatistics and Bioinformatics \\
    Roswell Park Comprehensive Cancer Center, Buffalo, NY \\ \\
    Fred Hutchinson Cancer Research Center, Seatlle, WA \\ \\
    CUNY Graduate School of Public Health, New York, NY}

\email{maintainer@bioconductor.org}
